\documentclass[../techreport.tex]{subfiles}
\begin{document}
\section{Specific Contribution 1}
\label{sec:specific1}
In this section, we \ldots \\

\noindent {\bf Reduction:} If graph $G = (V, E)$ contains a vertex $v \in V$ such that the closed neighborhood of $v$, denoted $N[v]$, is a clique, then $N[v]$ is in some minimum (vertex) clique cover and can be removed from $G$ to form reduced graph $G' = G[V \setminus N[v]]$, where $\theta(G) = 1 + \theta(G')$. \\ \\

\noindent {\bf Proposition:} The reduction is safe.

\begin{proof}
	Let $\mathcal{C} = \{C_1, C_2, \dots, C_k\}$ be a given minimal clique cover for $G$. Since $\mathcal{C}$ is a clique cover, there exists some clique in $\mathcal{C}$ that covers $v$. Let $C_v \in \mathcal{C}$ denote some clique such that $v \in C_v$. Observe that since we are given that the neighborhood of $v$, denoted $N(v)$, is a clique, by the definition of a clique, $N[v]$ is also a clique. Let $N[v]$ be denoted by $X$. \\

	If $X \in \mathcal{C}$, then $X$ is in a minimal clique cover, as desired. If $X \notin \mathcal{C}$, then we argue that we can swap $X$ with $C_v$ and maintain an optimal clique covering of $G$. \\

	Observe that, by the definition of a clique, $C_v$ consits only of $v$ and vertices $u \in N(v)$. Therefore, $C_v \subset X$. Hence, all $v \in C_v$ are covered by $X$. Let $\mathcal{C'} = (\mathcal{C} \setminus \{C_v\}) \cup \{X\}$ and notice that the single removal of $C_v$ and single addition of clique $X$ ensures that $|\mathcal{C'}| = |\mathcal{C}| = k$. Thus, we have a minimum clique cover $\mathcal{C'}$ such that $X \in \mathcal{C'}$. \\

	If $X = C_v$, then all $v \in X$ reside in a single clique $X$, as desired. If $X \neq C_v$, then there exists some $u \in N(v)$ such that $u \notin C_v$. Since $\mathcal{C}$ is a clique cover, then there exists some clique in $\mathcal{C}$ that covers $u$. Let $C_u \in C$ denote some clique such that $u \in C_u$. Then, by the definition of a clique, $C_u' = C_u \setminus X$ denotes some clique that covers all $u \in C_u$ and $u \notin X$. Since all $u \in X$ are covered by $X$ then we can swap $C_u$ with $C_u'$. Let $\mathcal{C''} = (\mathcal{C'} \setminus \{C_u\}) \cup \{C_u'\}$ and notice that the single removal of $C_u$ and single addition of clique $C_u'$ ensures that $|\mathcal{C''}| = |\mathcal{C'}| = k$. \\

	Observe that all $v \in X$ reside in a single clique $X$. Therefore, we can remove $X$ from $G$ to form reduced graph $G' = G[V \setminus N[v]]$, where $\theta(G) = 1 + \theta(G')$, as desired.
\end{proof}
%\subsection{Step 1 of Our Technique}
%\lipsum[16]
\end{document}
