%%
%% This is file `sample-sigconf.tex',
%% generated with the docstrip utility.
%%
%% The original source files were:
%%
%% samples.dtx  (with options: `sigconf')
%% 
%% IMPORTANT NOTICE:
%% 
%% For the copyright see the source file.
%% 
%% Any modified versions of this file must be renamed
%% with new filenames distinct from sample-sigconf.tex.
%% 
%% For distribution of the original source see the terms
%% for copying and modification in the file samples.dtx.
%% 
%% This generated file may be distributed as long as the
%% original source files, as listed above, are part of the
%% same distribution. (The sources need not necessarily be
%% in the same archive or directory.)
%%
%% The first command in your LaTeX source must be the \documentclass command.
\documentclass[sigconf]{acmart}

\newif\ifFull
\Fullfalse

\newif\ifDoubleBlind
%\DoubleBlindtrue
\DoubleBlindfalse

\newcommand{\Id}[1]{{\texttt{#1}}}

\let\acmcomment\comment
\let\comment\undefined

%\usepackage[final]{changes}
\usepackage{changes}
\definechangesauthor[name={Darren Strash}, color=blue]{DS}
\definechangesauthor[name={Louise Thompson}, color=green]{LT}

\let\mycomment\comment
\let\comment\acmcomment

\newcommand{\marrow}{\marginpar[\hfill$\longrightarrow$]{$\longleftarrow$}}
\newcommand{\niceremark}[3]{\textcolor{red}{\textsc{#1 #2: }}\textcolor{blue}{\textsf{#3}}}
\newcommand{\darren}[2][says]{\niceremark{Darren}{#1}{#2}}
\newcommand{\louise}[2][says]{\niceremark{Darren}{#1}{#2}}

%\renewcommand{\niceremark}[3]{}


\usepackage{textcomp}
\usepackage{amsmath}
\let\proof\relax
\let\endproof\relax
\usepackage{amsthm}
\usepackage{amsfonts}
\usepackage{algorithm}
\usepackage[noend]{algpseudocode}
\usepackage{mathtools}
\usepackage{wasysym}
\usepackage{array}
\usepackage{subcaption,siunitx,booktabs}
\usepackage{hyperref}
\usepackage{doi}
\usepackage{numprint}
\npdecimalsign{.} % we want . not , in numbers

\usepackage{color, colortbl}
\definecolor{Gray}{gray}{0.9}
\definecolor{darkblue}{RGB}{0, 0, 255}

\DeclarePairedDelimiter{\ceil}{\lceil}{\rceil}
% fixing lines ending on one word etc...
\setlength\parfillskip{0pt plus .4\textwidth}
\setlength\emergencystretch{.1\textwidth}
\clubpenalty10000
\widowpenalty10000
\displaywidowpenalty=10000

\usepackage{xspace}
\newcommand{\Is}       {:=}
\newcommand{\setGilt}[2]{\left\{ #1\sodass #2\right\}}
\newcommand{\sodass}{\,:\,}
\newcommand{\set}[1]{\left\{ #1\right\}}
\newcommand{\todoupdate}[1]{{\color{red}#1}}
\newcommand{\gilt}{:}
\newcommand{\pic}{}
\newcommand{\erdos}{Erd{\H o}s-R{\'e}nyi}

\def\MdR{\ensuremath{\mathbb{R}}}
\newcommand{\ie}{i.\,e.,\xspace}
\newcommand{\eg}{e.\,g.,\xspace}
\newcommand{\etal}{et~al.\xspace}

\usepackage{xspace}
\newcommand{\external}[0]{external\xspace}


\newcommand{\CC}[0]{{\ensuremath\mathcal{C}}}

\usepackage{enumerate}

%% Rights management information.  This information is sent to you
%% when you complete the rights form.  These commands have SAMPLE
%% values in them; it is your responsibility as an author to replace
%% the commands and values with those provided to you when you
%% complete the rights form.
\setcopyright{acmcopyright}
\copyrightyear{2019}
\acmYear{2019}
\acmDOI{}

%% These commands are for a PROCEEDINGS abstract or paper.
\acmConference[SIGMOD '20]{SIGMOD '20: ACM International Conference on Management of Data}{Jun 14--19, 2020}{Portland Oregon}
\acmBooktitle{SIGMOD '20: ACM International Conference on Management of Data,
  June 14--19, 2020, Portland, OR}
\acmPrice{}
\acmISBN{}


%%
%% Submission ID.
%% Use this when submitting an article to a sponsored event. You'll
%% receive a unique submission ID from the organizers
%% of the event, and this ID should be used as the parameter to this command.
%%\acmSubmissionID{123-A56-BU3}

%%
%% The majority of ACM publications use numbered citations and
%% references.  The command \citestyle{authoryear} switches to the
%% "author year" style.
%%
%% If you are preparing content for an event
%% sponsored by ACM SIGGRAPH, you must use the "author year" style of
%% citations and references.
%% Uncommenting
%% the next command will enable that style.
%%\citestyle{acmauthoryear}

%%
%% end of the preamble, start of the body of the document source.
\begin{document}

%%
%% The "title" command has an optional parameter,
%% allowing the author to define a "short title" to be used in page headers.
\title{Kernelization for the Clique Cover Problem}

\date{}

\pagestyle{plain}

\ifDoubleBlind
\author{}

%\authorrunning{Anonymous}
%\Copyright{Anonymous}
\else
\author{Darren Strash}
\authornote{Both authors contributed equally to this research.}
\email{dstrash@hamilton.edu}
\orcid{0000-0001-7095-8749}
\affiliation{%
  \institution{Department of Computer Science\\Hamilton College}
  \streetaddress{198 College Hill Road}
  \city{Clinton}
  \state{New York}
  \postcode{13323}
}
\author{Louise Thompson}
\authornotemark[1]
\email{lmthomps@hamilton.edu}
\affiliation{%
  \institution{Department of Computer Science\\Hamilton College}
  \streetaddress{198 College Hill Road}
  \city{Clinton}
  \state{New York}
  \postcode{13323}
}

\fi{}%

%%
%% By default, the full list of authors will be used in the page
%% headers. Often, this list is too long, and will overlap
%% other information printed in the page headers. This command allows
%% the author to define a more concise list
%% of authors' names for this purpose.
\renewcommand{\shortauthors}{Strash and Thompson}

%%
%% The abstract is a short summary of the work to be presented in the
%% article.
\begin{abstract}
We introduce an extensive collection of data reduction rules for the clique cover problem, showing that the clique cover problem has a linear kernel in the (vertex) cover number and implying it is fixed-parameter tractable on the cover number. Furthermore, we implement our new data reduction rules and show their efficacy in practice. Using our new data reduction rules, we are able to exactly compute exact minimum clique covers on graphs on millions of vertices. Of those graphs that we are unable to solve exactly with our reductions, we are able to solve XX \darren{fill in} by combining our technique with the heuristic algorithm by Chalupa [ACM Bulletin, Slovakia, 6(3) pp. 1--8, 2014]. 

Of further interest, our work provides a theoretical explanation for matching clique cover and independence numbers in practice, and furthermore, allows us to compute upper bounds of the independence number for huge instances that are not currently solved exactly.\darren{How close are we? Give a percentage, i.e., within 1\%?}
\end{abstract}

\maketitle

\begin{CCSXML}
\end{CCSXML}

\ccsdesc[500]{Computer systems organization~Embedded systems}
\ccsdesc[300]{Computer systems organization~Redundancy}
\ccsdesc{Computer systems organization~Robotics}
\ccsdesc[100]{Networks~Network reliability}

\keywords{minimum clique cover, data reductions, kernelization, fixed-parameter tractability}

\section{Introduction}
\label{sec:Introduction}
Given an undirected graph $G=(V,E)$, the \emph{clique cover problem}, also known as the clique partition problem~\cite{}, asks us to compute the smallest integer $k$ such that there exists a set of $k$ cliques $\mathcal{C} = \{C_1,C_2,\ldots, C_k\}$ where $\cup_{C\in\mathcal{C}} = V$. In other words, all vertices in the graph must be \emph{covered} by some clique. The smallest such $k$ is called the \emph{clique cover number} and is denoted by $\theta(G)$. 
The clique cover problem is a fundamental problem that has long been known to be NP-hard: Its decision variant is one of Karp's 21 original NP-complete problems~\cite{karp1972reducibility}. Furthermore, it is known to be hard to approximate within $\varepsilon$~\cite{}. Like many NP-hard problems, it has applications. Clique covers are \darren{list application areas} and can be further used as upper bounds for pruning branch-and-bound search for independent sets~\cite{}.

Expanding on work by Gramm et al. for \emph{edge} clique covers, Mujuni and Rosamond~\cite{mujuni-2008} introduced simple data reduction rules for the clique clique partition problem, proving the problem has a kernel of size $k^2$, where $k$ is the clique cover number. This result further gives a fixed-parameter tractable algorithm for the problem. That is, their algorithm has running time $O(f(k)n^c)$, where $f$ is a computable function and $c$ is a constant independent of $k$.

Curiously, recent experiments by Chalupa show that the independence number and clique cover number match for many small sparse real-world graphs. There is currently no theoretical explanation for this, as it is well known that the independent number and clique cover number can be arbitrarily far apart~\cite{}.

\subsection{Our Results.}
\darren{Insert variation based on abstract; also note that we transfer many reductions designed for the vertex cover problem.}

\section{Preliminaries}
\label{sec:Preliminaries}
We work with a finite unweighted undirected graph $G=(V,E)$ with $n = |V|$ vertices and $m=|E|$ edges. The \emph{open neighborhood} of a vertex $v\in V$ is defined to be $N(v)=\{u\mid \{u,v\}\in E\}$, and the \emph{closed neighborhood} is defined to be $N[v] = N(v)\cup\{v\}$. The \emph{degree} of a vertex $v\in V$, is defined to be $d(v) = |N(v)|$.
A \emph{clique}, also called a complete subgraph, is a set of vertices $C\subseteq V$ such that $\forall u,v\in C\,\,\{u,v\}\in E$. We say that a set of cliques $\CC=\{C_1,C_2,\ldots, C_k\}$ \emph{covers} the vertices graph $G$ if $\cup_{C\in\CC}C = V$ and $\CC$ is called a \emph{clique cover}. We denote by $\theta(G)$ the \emph{clique cover number}, the minimum cardinality of any clique cover of $G$.

A \emph{clique partition} is a disjoint partition of the vertices of $V$ into cliques: that is, a clique cover $\CC=\{C_1,C_2,\ldots,C_k\}$ such that for $i\neq j$ $C_i\cap C_j\neq \emptyset$. Given any clique cover of $G$, a clique partition of the same size can be obtained. Hence, the clique cover problem is at times called the clique partition problem.

Tightly connected to the clique cover problem is the \emph{independent set problem}, the NP-hard problem whose goal is to compute a maximum number of independent vertices --- that is, vertices that do not share an edge. This number is called the independence number, denoted $\alpha(G)$. It is well known that $\alpha(G)\subseteq \theta(G)$, since a clique can contain at most one vertex of an independent set.

Furthermore, we note that graph coloring problem is complementary to the problem considered here. A valid coloring is a partition of the graph into independent sets, each one being a color class. Thus, a valid coloring of the complement graph $\overline{G}$, is a clique cover of $G$, hence the chromatic number $\gamma(\overline{G})$ is equivalent to $\theta(G)$. Indeed, this relationship is exploited a number of ways in the literature. Chalupa~\cite{} uses techniques from graph coloring to compute clique covers. The relationship is further borne out in the use of these problems in pruning for branch-and-bound solvers: graph coloring is used to prune search for cliques~\cite{}, and clique covers are used to prune search for independent sets~\cite{}.
\section{Related Work.}
\subsection{Clique Covers}
\subsection{Graph Coloring}
\subsection{Edge Clique Covers}
\subsection{Data Reduction Rules and Fixed-Parameter Tractability}
\label{sec:RelatedWork}


\section{New Data Reduction Rules for Minimum Vertex Clique Cover}
\section{Implementation Details}
\section{Experiments}
\subsection{Experimental Setup.}
\subsection{Data Sets.}
\subsection{Exactly Solved Instances}
\subsection{Speeding up with Heuristic Search}
\subsection{Checking the Quality of Independent Sets}

\bibliographystyle{ACM-Reference-Format}
\bibliography{references}
\end{document}

\endinput
%%
%% End of file `sample-sigconf.tex'.
